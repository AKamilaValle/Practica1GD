
\documentclass[letterpaper,11pt]{article}
%%%%%%%%%%%%%%%%%%%%%%%%%%%%%%%%%%%%%%%%%%%%%%%%%%%%%%%%%%%%%%%%%%%%%%%%%%%%%%%%%%%%%%%%%%%%%%%%%%%%%%%%%%%%%%%%%%%%%%%%%%%%%%%%%%%%%%%%%%%%%%%%%%%%%%%%%%%%%%%%%%%%%%%%%%%%%%%%%%%%%%%%%%%%%%%%%%%%%%%%%%%%%%%%%%%%%%%%%%%%%%%%%%%%%%%%%%%%%%%%%%%%%%%%%%%%
\usepackage{graphicx}
\usepackage{amsmath,amsfonts,amssymb,amsthm,float}
\usepackage{hyperref}
\usepackage[utf8]{inputenc}
\usepackage[left=2cm, right=2cm, top=2cm, bottom=2cm]{geometry}

\setcounter{MaxMatrixCols}{10}
%TCIDATA{OutputFilter=LATEX.DLL}
%TCIDATA{Version=5.50.0.2953}
%TCIDATA{<META NAME="SaveForMode" CONTENT="1">}
%TCIDATA{BibliographyScheme=BibTeX}
%TCIDATA{LastRevised=Wednesday, February 18, 2026 18:50:57}
%TCIDATA{<META NAME="GraphicsSave" CONTENT="32">}

\input{tcilatex}
\renewcommand{\baselinestretch}{1.15}
\setlength{\parindent}{0pt}
\setlength{\parskip}{0.5\baselineskip}
\pretolerance=2000 \tolerance=3000
\renewcommand{\abstractname}{Resumen}

\begin{document}

\title{Pr\'{a}ctica 1: Sistema Lotka-Volterra}
\author{Ana Kamila Valle Z. Flores $\left[ 22211769\right] $ \\
%EndAName
Departamento de Ingenier\'{\i}a El\'{e}ctrica y Electr\'{o}nica\\
Tecnol\'{o}gico Nacional de M\'{e}xico / Instituto Tecnol\'{o}gico de Tijuana%
}
\maketitle

\noindent \textbf{Palabras clave: }Palabra clave 1; Palabra clave 2; Palabra
clave 3; Palabra clave 4; Palabra clave 5.

\noindent El autor debe identificar palabras clave espec\'{\i}ficas que
representen con precisi\'{o}n la investigaci\'{o}n, se debe evitar usar t%
\'{e}rminos generales o vagos que podr\'{\i}an aplicarse a muchos temas
diferentes; se sugiere escribir de cuatro a siete palabras clave ordenadas
alfab\'{e}ticamente, separadas por punto y coma, y omitir palabras que ya est%
\'{a}n en el t\'{\i}tulo del documento.

\bigskip

\noindent Correo: \textbf{l22211769@tectijuana.edu.mx}

\noindent \noindent Carrera: \textbf{Ingenier\'{\i}a Biom\'{e}dica }

\noindent Asignatura: \textbf{Biolog\'{\i}a de Sistemas}

\noindent Profesor: \href{https://biomath.xyz/}{\textbf{Dr. Paul Antonio
Valle Trujillo}} (paul.valle@tectijuana.edu.mx)

\section{Modelo matem\'{a}tico}

El sistema presa-depredador de Lotka-Volterra se formula mediante las
siguientes dos Ecuaciones Diferenciales Ordinarias (EDOs) no lineales de
primer orden:

\begin{eqnarray*}
\dot{x} &=&\alpha x-\beta xy, \\
\dot{y} &=&\delta xy-\gamma y,
\end{eqnarray*}
donde $\alpha ,\beta ,\delta ,\gamma >0$ y las condiciones iniciales $%
x\left( 0\right) ,y\left( 0\right) \geq 0$. La variable $x\left( t\right) $
representa a las presas [\textit{hares}], la variabele $y\left( t\right) $%
representa a los depredadores [\textit{lynx}] y el tiempo $t$ se mide en a%
\~{n}os.

\section{An\'{a}lis de positividad}

El an\'{a}lisis de positividad permite obtener conclusiones sobre la din\'{a}%
mica de las semitrayectorias positivas $\left( \Gamma ^{+}\right) $ [secci%
\'{o}n 2.3] para condiciones iniciales no negativas $[x\left( 0\right)
,y\left( 0\right) \geq 0]$, para esto se debe evaluar al sistema en la
frontera del primer cuadrante (ortante positivo) como se indica a continuaci%
\'{o}n:

\begin{eqnarray*}
\left. \dot{x}\right\vert _{x=0} &=&\alpha \left( 0\right) -\beta \left(
0\right) y=0, \\
\left. \dot{y}\right\vert _{y=0} &=&\delta x\left( 0\right) -\gamma \left(
0\right) =0,
\end{eqnarray*}%
por lo tanto, con base en el \textbf{Lema de Positividad }en sistemas no
lineales [secci\'{o}n 2.6], se establece que el sistema presa-depredador de
Lotka-Volterra tiene soluciones psoitivamente invariantes [secci\'{o}n 3.1]
para condiciones iniciales no negativas.

\section{Puntos de equilibrio}

Para calcular los puntos de equilibrio es necesario igualar cada ecuaci\'{o}%
n del sistema cero y resolver para cada una de las variables, es decir,%
\begin{eqnarray*}
\alpha x-\beta xy &=&0, \\
\delta xy-\gamma y &=&0,
\end{eqnarray*}

$\func{assume}\left( \alpha ,\func{positive}\right) =\left( 0,\infty \right) 
$

$\func{assume}\left( \beta ,\func{positive}\right) =\left( 0,\infty \right) $

$\func{assume}\left( \delta ,\func{positive}\right) =\left( 0,\infty \right) 
$

$\func{assume}\left( \gamma ,\func{positive}\right) =\left( 0,\infty \right) 
$

lo anterior permite establecer que los cuatro par\'{a}metros del sistema son
positivos. Al aplicar el comando compute-solve-exact, se obtienen los
siguientes puntos de equilibrio: 
\begin{eqnarray*}
\left( x_{1}^{\ast },y_{1}^{\ast }\right) &=&\left( 0,0\right) , \\
\left( x_{1}^{\ast },y_{1}^{\ast }\right) &=&\left( \frac{\gamma }{\delta },%
\frac{\alpha }{\beta }\right) ,
\end{eqnarray*}%
Nota: En algunos libros cuando el origen del espacio de estados es un
quilibrio del sistema, se escribe asi: 
\begin{equation*}
\left( x_{1}^{\ast },y_{1}^{\ast }\right) =\left( 0,0\right) .
\end{equation*}

\section{Estabilidad asint\'{o}tica}

Para analizar estabilidad local de los puntos de equlibrio de un sistema din%
\'{a}mico [secciones 5.1 y 5.3] es encesario calcular la matriz jacobiana
[secci\'{o}n 2.7], como se indica a continuaci\'{o}n:%
\begin{equation*}
J=\left[ 
\begin{array}{cc}
\frac{\partial \dot{x}}{\partial x} & \frac{\partial \dot{x}}{\partial y} \\ 
\frac{\partial \dot{y}}{\partial x} & \frac{\partial \dot{y}}{\partial y}%
\end{array}%
\right] =\left[ 
\begin{array}{cc}
\alpha -\beta y & -\beta x \\ 
\delta y & \delta x-y%
\end{array}%
\right] 
\end{equation*}%
Ahora, se evaluacada uno de los puntos de equilibrio calculados, primero se
analiza $\left( x_{1}^{\ast },y_{1}^{\ast }\right) =\left( 0,0\right) ,$
obteniendo el siguiente resultado%
\begin{equation*}
\left. J\right\vert _{\left( x_{1}^{\ast },y_{1}^{\ast }\right) }=\left[ 
\begin{array}{cc}
\alpha -\beta y & -\beta x \\ 
\delta y & \delta x-y%
\end{array}%
\right] _{x=0,y=0}=\left[ 
\begin{array}{cc}
\alpha  & 0 \\ 
0 & -\gamma 
\end{array}%
\right] ,
\end{equation*}%
entonces, debido a que el resultado es una matriz diagonal [secci\'{o}n
2.7], los valores propios est\'{a}n dados por cada uno de los elementos de
dicha diagonal, es decir,%
\begin{eqnarray*}
\lambda _{1} &=&\alpha , \\
\lambda _{2} &=&-\gamma ,
\end{eqnarray*}%
con base en estos elementos se concluye lo siguiente: El punto de equilibrio 
$\left( x_{1}^{\ast },y_{1}^{\ast }\right) =\left( 0,0\right) $ es inestable
[secciones 5.1 y 5.2.3].

Al evaluar el segundo equilibrio $\left( x_{2}^{\ast },y_{2}^{\ast }\right)
=\left( \frac{\gamma }{\delta },\frac{\alpha }{\beta }\right) $, se obtiene
el siguiente resultado%
\begin{equation*}
\left. J\right\vert _{\left( x_{2}^{\ast },y_{2}^{\ast }\right) }=\left[ 
\begin{array}{cc}
\alpha -\beta y & -\beta x \\ 
\delta y & \delta x-y%
\end{array}%
\right] _{x=\frac{\gamma }{\delta },y=\frac{\alpha }{\beta }}=\left[ 
\begin{array}{cc}
\alpha -\beta \frac{\alpha }{\beta } & -\beta \frac{\gamma }{\delta } \\ 
\delta \frac{\alpha }{\beta } & \delta \frac{\gamma }{\delta }-\gamma 
\end{array}%
\right] =\left[ 
\begin{array}{cc}
0 & -\beta \frac{\gamma }{\delta } \\ 
\delta \frac{\alpha }{\beta } & 0%
\end{array}%
\right] ,
\end{equation*}%
por lo tanto , se deben calcular los valores propios mediante la siguiente
ecuaci\'{o}n [secci\'{o}n 2.7]:%
\begin{equation*}
\det \left( A-\lambda I\right) =0,
\end{equation*}%
donde $A$ es la matriz Jacobiana evaluada en el punto de equilibrio, $%
\lambda \epsilon \mathbf{R}$ (es un coeficiente) e $I$ es la matriz
identidad, al sustituir se obtiene lo siguiente: 
\begin{equation*}
\det \left[ \left( 
\begin{array}{cc}
0 & -\frac{\beta \gamma }{\delta } \\ 
\frac{\delta \alpha }{\beta } & 0%
\end{array}%
\right) -\lambda \left( 
\begin{array}{cc}
1 & 0 \\ 
0 & 1%
\end{array}%
\right) \right] =0,
\end{equation*}%
entonces, al realizar las operaciones aritm\'{e}ticas necesarias se obtiene
el siguiente resultado:%
\begin{eqnarray*}
\det \left[ \left( 
\begin{array}{cc}
0 & -\frac{\beta \gamma }{\delta } \\ 
\frac{\delta \alpha }{\beta } & 0%
\end{array}%
\right) -\left( 
\begin{array}{cc}
\lambda  & 0 \\ 
0 & \lambda 
\end{array}%
\right) \right]  &=&0, \\
\det \left( 
\begin{array}{cc}
-\lambda  & -\frac{\beta \gamma }{\delta } \\ 
\frac{\delta \alpha }{\beta } & -\lambda 
\end{array}%
\right)  &=&0, \\
\lambda ^{2}+\alpha \gamma  &=&0,
\end{eqnarray*}%
y valores propios est\'{a}n dados por%
\begin{eqnarray*}
\lambda _{1} &=&0+\sqrt{-\alpha \gamma }=+i\sqrt{\alpha \gamma }, \\
\lambda _{2} &=&0-\sqrt{-\alpha \gamma }=-i\sqrt{\alpha \gamma },
\end{eqnarray*}%
y se concluye que el punto de equilibrio es marginalmente estable,
denominado centro [secci\'{o}n 5.2.3], por lo tanto, en una vencidad
suficiententemente cerca del punto de equilibrio se esperan oscilasciones
sostenidas a lo largo del tiempo $\left( t\geq 0\right) .$

\end{document}
